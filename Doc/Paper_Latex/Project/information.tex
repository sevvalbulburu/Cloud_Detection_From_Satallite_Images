%%%%%%%%%%%%%%%%%%%%%%%%%%%%%%%%%%%%%%%%%%%%%%%%%%%%%%%%%%%%%%%%%%%%%%%
%%%%%%%%% Aşağıda istenilen bilgileri dikkatlice doldurunuz.   %%%%%%%%
%%%%%%%%% Doldurmanız istenilen ifadenin sonunda TR ya da EN   %%%%%%%%
%%%%%%%%% yazıyorsa, sırasıyla Türkçe veya İngilizce olarak    %%%%%%%%
%%%%%%%%% doldurunuz. Eğer herhangi bir ifade yoksa, projenizi %%%%%%%%
%%%%%%%%% hangi dilde yazıyorsanız (Türkçe veya İngilizce), o  %%%%%%%%
%%%%%%%%% dile göre doldurunuz. İsimleri yazarken soyisimleri  %%%%%%%%
%%%%%%%%% büyük harf ile yazınız.                              %%%%%%%%
%%%%%%%%%%%%%%%%%%%%%%%%%%%%%%%%%%%%%%%%%%%%%%%%%%%%%%%%%%%%%%%%%%%%%%%
%%%%%%%%%%%%%%%%%%%%%%%%%%%%%%%%%%%%%%%%%%%%%%%%%%%%%%%%%%%%%%%%%%%%%%%


% Proje başlığını Türkçe olarak yazınız.
\def\titleTR{ UYDU GÖRÜNTÜLERİNDEN BULUT TESPİTİ }

% Proje başlığını İngilizce olarak yazınız.
\def\titleEN{ CLOUD DETECTION IN SATELLITE IMAGES }

% Proje grubundaki ilk ismi yazınız.
\def\studenti{Mehmet Alperen ÖLÇER}
%İlk öğrencinin, öğrenci numarasını yazınız.
\def\numberi{20011023}
% Proje grubundaki ilk öğrencinin doğum tarihi ve yerini yazınız.
\def\studentibdate{19.07.2001, Balikesir}
% Proje grubundaki ilk öğrencinin e-mail adresini yazınız.
\def\studentiemail{alperen.olcer@std.yildiz.edu.tr}
% Proje grubundaki ilk öğrencinin cep telefonu numarasını yazınız.
\def\studentiphone{ 0553 841 79 50}
% Proje grubundaki ilk öğrencinin staj deneyimlerini yazınız. Satır atlatmak için \\ kullanabilirsiniz.
\def\studentiintern{ During my internship at Turkcell, I received hands-on experience\\
in software development and DevOps by working on a Jira plugin in Java and SQL.\\
This experience helped me improve my abilities and knowledge of the sector.}

% Proje grubundaki ikinci ismi yazınız. Eğer ikinci üye yoksa ~ işareti ekleyiniz ve ikinci
% öğrenci ile alakalı diğer bilgileri atlayınız.
\def\studentii{Şevval BULBURU}
%İkinci öğrencinin, öğrenci numarasını yazınız.
\def\numberii{19011038}
% Proje grubundaki ikinci öğrencinin doğum tarihi ve yerini yazınız.
\def\studentiibdate{02.03.2000, Istanbul}
% Proje grubundaki ikinci öğrencinin e-mail adresini yazınız.
\def\studentiiemail{sevval.bulburu@std.yildiz.edu.tr}
% Proje grubundaki ikinci öğrencinin cep telefonu numarasını yazınız.
\def\studentiiphone{ 0541 840 39 60}
% Proje grubundaki ikinci öğrencinin staj deneyimlerini yazınız. Satır atlatmak için \\ kullanabilirsiniz.
\def\studentiiintern{I worked on developing game engine and GUI elements with C++ at Nitra Game Software.}

% Projeyi teslim ettiğiniz ay ve yılı proje için kullandığınız dilde yazınız.
\def\date{May, 2023}

% Proje danışmanınızın ismini Türkçe ünvanı ile yazınız.
\def\advisorTR{Prof. Dr. Mine Elif KARSLIGİL}
% Proje danışmanınızın ismini İngilizce ünvanı ile yazınız.
\def\advisorEN{Prof. Dr. Mine Elif KARSLIGİL}

\def\acknowledgementText{
    % Buraya teşekkür metninizi proje için kullandığınız dilde yazınız. 
Yildiz Technical University, founded in 1911 in Istanbul, Turkey, is a prominent government institution with 10 faculties, 2 institutes, and approximately 25,000 students. It is also regarded as one of the greatest in the country.

We'd like to thank our instructor Mine Elif Karsligil for her significant assistance, continuous follow-up, and direction throughout my thesis on "Cloud Detection in Satellite Images." Her knowledge of computer science, computer vision, algorithms, computer learning, artificial intelligence, image processing, pattern recognition, neural networks, technology and engineering was important in the project's success.
}

\def\abstractTextEnglish{
    % Buraya İngilizce olarak proje özetini yazınız.
Humans have attempted to fully understand and explain nature throughout history. Weather events may now be forecast owing to technological advancements. This is primarily made possible by tracking the clouds.

Cloud detection uses image processing technology that detects and analyzes clouds in satellite pictures. This method is utilized in meteorology, environmental monitoring, agriculture, forestry, urban planning, military intelligence, and a variety of other fields. Cloud detection technology is given by software that scans satellite photos to assess cloud features such as density, height, size, movement, and kind. This technology enables the collection of data for a variety of reasons, including environmental monitoring and natural catastrophe management.

Some satellite imagery studies make use of data collected from the Earth's surface. Clouds in the atmosphere, on the other hand, can impede the efficiency of such investigations when they are between the surface and the satellite. The primary purpose of this research is to solve such issues and enable the usage of cloud detection in applications. As a solution to this challenge, the U-Net architecture for the model, was adopted. The achieved binary accuracy of \%91, obtained after training, can be interpreted as a measure of success in the model's performance. 
}


\def\abstractKeywordsEnglish{
    % Buraya İngilizce olarak proje için geçerli anahtar kelimeleri yazınız
    Cloud detection, Image processing, satellite imagery, meteorology, environmental monitoring, data collection, U-Net architecture, semantic segmentation,  Landsat 8.
}

\def\abstractTextTurkish{
    % Buraya Türkçe olarak proje özetini yazınız
İnsanoğlu geçmişten günümüze kadar doğayı anlama ve analiz etme çabasında olmuştur. Gelişen teknoloji ile hava olayları tahmin edilebilir bir konuma gelmiştir. Bu durumu başlıca bulutların hareketlerini takip edebilmek mümkün hale getirmiştir.

Bulut tespiti, uydu görüntülerinde bulutların tespit edilmesi ve özelliklerinin analiz edilmesi için görüntü işleme teknikleriyle yapılır. Bu teknik, meteoroloji, çevre izleme, tarım, ormancılık, kent planlaması, askeri istihbarat ve diğer birçok uygulama alanında kullanılmaktadır. Bulut tespiti teknolojisi, uydu görüntülerini işleyen yazılımlar aracılığıyla sağlanır ve bulutların yoğunluğu, yüksekliği, boyutu, hareketi ve tipi gibi özelliklerini belirlemek için çeşitli algoritmalar kullanılır. Bu teknoloji, farklı amaçlara hizmet eden veriler elde edilmesine olanak tanır ve çevre izleme, doğal afet yönetimi gibi alanlarda da kullanılır.

Uydu görüntülerinde yapılan bazı çalışmalarda yeryüzünden toplanan verilerle çalışılır. Fakat atmosferdeki bulutlar yeryüzünden toplanacak veriler ile uydu arasında olduğundan, çalışmaların efektif olmasına engel olmaktadır. Yapılan projenin temel hedefi bu gibi sorunlara çözüm olmakla birlikte bulut tespiti ile gerçeklenebilecek uygulamalarda kullanımı sağlamaktır. İlgili soruna çözüm olarak U-Net mimarisi ile oluşturulmuş CNN ağı ile model oluşturulmuştur. Eğitimden sonra elde edilen \%91 doğruluk oranı, modelin performansındaki başarının bir ölçüsü olarak yorumlanabilir.

}

\def\abstractKeywordsTurkish{
    % Buraya Türkçe olarak proje için geçerli anahtar kelimeleri yazınız
    Bulut tespiti, Görüntü işleme, Uydu görüntüleri, Meteoroloji, Çevre izleme, Veri toplama, U-Net mimarisi, semantik segmentasyon, Landsat 8.
}

% Proje için gerekli olan sistem ve yazılım bilgilerini yazınız.
\def\software{ Jupyter Notebook, Python3, Tensorflow, Keras, Hub }

% Proje için gerekli olan RAM bellek boyutunu yazınız.
\def\memorysize{32GB}

% Proje için gerekli olan harddisk boyutunu yazınız.
\def\disksize{32GB}

%%%%%%%%%%%%%%%%%%%%%%%%%%%%%%%%%%%%%%%%%%%%%%%%%%%%%%%%%%%%%%
%%%% Aşağıdaki alana "\item[sembol] Sembol_açıklaması" %%%%%%%
%%%% şeklinde sembollerinizi giriniz. Açıklamanın ilk  %%%%%%%
%%%% harfine göre sıralayınız. Eğer sembol kullanmı-   %%%%%%%
%%%% yorsanız "\def\symbols{}" olacak şekilde küme     %%%%%%%
%%%% parantezlerinin içini siliniz.                    %%%%%%%
%%%%%%%%%%%%%%%%%%%%%%%%%%%%%%%%%%%%%%%%%%%%%%%%%%%%%%%%%%%%%%

\def\symbols{

}

%%%%%%%%%%%%%%%%%%%%%%%%%%%%%%%%%%%%%%%%%%%%%%%%%%%%%%%%%%%%%%
%%%% Aşağıdaki alana "\item[kısaltma] kısaltma_açıklaması" %%%
%%%% şeklinde kısaltmalarınızı giriniz. Kısaltmanın ilk    %%%
%%%% harfine göre sıralayınız. Eğer kısaltma kullanmıyor-  %%%
%%%% sanız "\def\abbrevations{}" olacak şekilde küme pa-   %%%
%%%% rantezlerinin içini siliniz.                          %%%
%%%%%%%%%%%%%%%%%%%%%%%%%%%%%%%%%%%%%%%%%%%%%%%%%%%%%%%%%%%%%%
\def\abbrevations{

    \begin{abbrv}
        \item[RS-Net ]          Remote Sensing Network
        \item[CNN]              Convolutional Neural Network
        \item[L8]               Landsat 8
        \item[L7]               Landsat 7
        \item[IOU]              Intersection over union
        \item[CCA]              Cloud Cover Assessment
        \item[NSS]              Natural scene statistic
        \item[F1]               F Score
        \item[SVM]              Support Vector Machine 
        \item[SLIC]             Simple Linear Iterative Clustering
    \end{abbrv}
    
}