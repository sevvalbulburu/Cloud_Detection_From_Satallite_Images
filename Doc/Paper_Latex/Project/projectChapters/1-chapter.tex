\chapter{Introduction}
A cloud is an apparent collection of atmospherically condensed water vapor that is normally floating far above the level of the earth. Clouds are crucial because they hand over rain, but they also get in the way of satellite photos' attempts to study the earth's surface. Since clouds on the input image are considered noise, detecting and removing clouds from satellite images is a crucial preprocessing step for the majority of remote sensing applications. The Earth's temperature, the dynamics of the terrestrial atmosphere, the thermodynamic chemistry, and radiative transport are all influenced by clouds. In addition to these, after the Kahramanmaras earthquake on February 6, 2023, the detection of the collapsed areas could not be realized in real-time using satellite images. The reason for this is that clear images cannot be obtained from the ground due to the dust clouds caused by the debris and snowfall. The difficulties encountered have hindered the timely delivery of the necessary aid by making it difficult to identify the areas that have been destroyed. It is anticipated that this situation can be overcome with a design that can detect clouds in the affected areas. Satellite images are one of the most powerful and important tools used by the scientist for the study of earth and space science.

In the past few years, numerous satellite imaging resources have been employed by researchers to examine and monitor the Earth's atmosphere. These resources encompass the MODIS data, the Greenhouse Gas Observing Satellite (GOSAT), Landsat-8, a U.S. Earth observation satellite, France's Sentinel-1/2 satellites, the Spot series of satellites from Europe, the Indian Ocean Infrared Satellite FY2G, China's high-resolution Earth observation satellite GaoFen-1, and the GOSAT. Furthermore, geostationary weather satellites such as Landsat-8, GOES, INSAT-3D/3DR, Kalpana-1, and the Meteosat series from the Indian Space Research Organization, the NOAA Advanced Very High-Resolution Radiometer, the QuickBird satellite, and the Ikonos satellites have also been used. These satellites, positioned high above the Earth, offer invaluable perspectives on cloud movements and other atmospheric events.\cite{mahajan2020cloud}

Cloud detection from satellite imagery can be difficult due to many kind of causes. The complexity of cloud patterns, which can vary substantially in size, shape, and texture, is one of the key challenges. Clouds can also overlap with other terrain features such as mountains, coastlines, or forests, making detection even more difficult. Another challenge in detecting clouds is the presence of other meteorological phenomena such as haze, dust, or smoke, which can mimic the appearance of clouds and generate false detections. Furthermore, clouds can have different optical properties at different times of day and in different weather conditions, making detection even more difficult.

In conclusion, cloud detection has many challenges despite of its importance. This project is an alternative solution to cloud detection from satellite images. Semantic segmantation based on U-Net is used for classification and  a cloud identification model was trained in the scope of the research by including extreme cases in the dataset obtained from Landsat-8 satellite pictures utilized in the project.